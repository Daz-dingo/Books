\documentclass[12pt,heading=true]{book}
\usepackage[UTF8,scheme=chinese]{ctex}
\usepackage{geometry}
\geometry{
    papersize={6in,9in},
    total={4.7in,7.4in},
    left=0.5in,
    top=0.6in,
}
\usepackage{indentfirst}
\usepackage{etoolbox}
\usepackage{relsize}
\usepackage{subfiles}
\CTEXsetup[name={第,节},number={\chinese{section}}]{section}
\CTEXsetup[name={第,卷},number={\chinese{part}}]{part}

\begin{document}
\title{权力,财富与社会}
\author{叶子晟,曾广智\ 著}
\date{}
\maketitle

\tableofcontents

% ====
% part
% ====
% title: 权力
% name: Zisheng, Ye
% data: 09/08/2019
% content: 分析权力的来源与性质;讨论不同规模群体活动中的权力架构组织模式;讨论在小规模(小团队)与中等规模(中等企业)群体活动中权力的分配与集中;讨论权力在同质群体(如技术团队)和异质群体(如技术与销售组成的公司团队)中权力的产生、分配、争夺与摩擦;
% =============
% end of a part
% =============
\part{权力}

% =======
% chapter
% =======
% title: 权力的来源与性质
% name: Zisheng, Ye
% date: 09/08/2019
% content: 讨论权力的群体性、动物性来源;简述权力在群体性活动中的重要性、不可或缺性及其带来的效率;
% ================
% end of a chapter
% ================
\subfile{CH权力的来源与性质/chapter.tex}

% =======
% chapter
% =======
% title: 权力的集中
% name: Zisheng, Ye
% date: 09/08/2019
% content: 讨论权力集中的不可避免性;权力集中对于决策效率的影响;同领域中多个群体不同权力集中度下的效率,短期与长期效果;
% ================
% end of a chapter
% ================
\subfile{CH权力的集中/chapter.tex}

% ====
% part
% ====
% title: 财富
% name: Zisheng, Ye
% data: 09/08/2019
% content: 分析财富的来源与性质;讨论财富的产生、消耗与再生;
% =============
% end of a part
% =============
\part{财富}

% ====
% part
% ====
% title: 社会中的权力与财富
% name: Zisheng, Ye
% data: 09/08/2019
% content: 讨论权力与财富在社会生活的传递、转移与演化;讨论权力与财富在社会生活中的相互依存性;
% =============
% end of a part
% =============
\part{社会中的权力与财富}

% =======
% chapter
% =======
% title: 代际传承
% name: Zisheng, Ye
% date: 09/08/2019
% content: 讨论权力与财富在代际之间传承的方式、过程以及稳定传承与破坏传承的方法与结果;讨论权力与财富代际传承的公平性、意义与价值;
% ================
% end of a chapter
% ================
\subfile{CH代际传承/chapter.tex}
\end{document}
% !TEX root = ./main.tex

\documentclass[main.tex]{subfiles}
 
\begin{document}
\chapter{权力的来源与性质}
这一章,我们希望透过对群体性行为的研究来分析权力的多种来源以及不同来源下的权力所拥有的不同性质。在我们的认知中,权力应当是一种群体性活动的产物,对于孤立个体来说,权力并不存在且没有存在的意义。孤立的个体,仅能对其个人产生影响,而权力则是一个个体或者一个群体对于另一个个体或者群体产生作用的桥梁。权力的存在与群体性的决策紧密关联。权力属于认知过程中调动其他个体或群体的驱动力以及在群体里执行任务的连接器。

\section{群体的决策}
在心理学中,决策是一种认知过程。权力是一种社群活动的产物,因此我们仅考虑群体性的决策对于权力的影响。群体性的决策来源于个体与个体决策的平衡与妥协。个体与个体的决策因为个体认知能力的差异,会导致不同的决策结果。然而群体中

\section{个体的差异性}
我们必须承认个体存在差异性。这种差异性我们不应假设其存在高下之分,所有的差异性都应当等同视之。

\section{来源于动物性的权力}
权力拥有强烈的动物性来源,许多动物拥有与人类极为类似的社会模式。
\end{document}
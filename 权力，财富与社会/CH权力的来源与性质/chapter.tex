% !TEX root = ../main.tex

\documentclass[main.tex]{subfiles}
 
\begin{document}
\chapter{权力的来源与性质}
这一章,我们希望透过对群体性行为的研究来分析权力的多种来源以及不同来源下的权力所拥有的不同性质。在我们的认知中,权力应当是一种群体性活动的产物,对于孤立个体来说,权力并不存在且没有存在的意义。孤立的个体,仅能对其个人产生影响,而权力则是一个个体或者一个群体对于另一个个体或者群体产生作用的桥梁。权力的存在与群体性的决策紧密关联。权力属于认知过程中调动其他个体或群体的驱动力以及在群体里执行任务的连接器。

% =====
% issue
% =====
% title: 节间顺序问题
% name: Zisheng, Ye
% date: 09/08/2019
% content: 需要考虑逻辑上,应当先从个体分析起,还是从群体分析起。这个区别应当影响有关于『群体的决策』以及『个体的差异性』这两节的顺序关系。
% ===============
% end of an issue
% ===============

% =======
% section
% =======
% title: 群体的决策
% name: Zisheng, Ye
% date: 09/08/2019
% content: 群体如何决策,权力如何在群体决策过程中产生
% ================
% end of a section
% ================
\subfile{CH权力的来源与性质/Sc群体的决策.tex}
% =====
% issue
% =====
% title: 为何需要这一节
% name: Zisheng, Ye
% date: 09/08/2019
% content: 我们应当认为,权力来源于群体活动,单独个体没有权力的概念,分析群体活动中如何决策,以及赋予群体中部分个体或全体以参与最终决策的意义,以及参与决策与权力的关系
% ===============
% end of an issue
% ===============

% =======
% section
% =======
% title: 个体的差异性
% name: Zisheng, Ye
% date: 09/08/2019
% content: 个体的差异如何导致决策的偏差,以及个体的差异性如何决定权利的分配与集中
% ================
% end of a section
% ================
\subfile{CH权力的来源与性质/Sc个体的差异性.tex}

% =======
% section
% =======
% title: 来源于动物性的权力
% name: Zisheng, Ye
% date: 09/08/2019
% content: 权力的概念不应仅存在于人类活动中,而是拥有强烈的动物性来源;阐述权力在于人类活动中的根本性地位
% ================
% end of a section
% ================
\subfile{CH权力的来源与性质/Sc来源于动物性的权力.tex}
\end{document}